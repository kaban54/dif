\documentclass[12pt,a4paper,fleqn]{article}
\usepackage[utf8]{inputenc}
\usepackage[russian]{babel}
\usepackage{graphicx}
\begin{document}
\begin{center}
{\Large \bf Методы дифференцирования функций в математическом анализе.}
\end{center}
Рассмотрим функцию
$$x^{2} $$Упростим выражение\\
$$x^{2} $$\\
Очевидно, данное выражение не нуждается в упрощении.\\
\newline\begin{figure}[h]
\centering
\includegraphics[width=0.8\linewidth]{images/plotimg136.png}
\caption{График функции}
\end{figure}
\newline
\newpage
\begin{center}
{\large \bf Нахождение производной.}
\end{center}
Продифференцируем выражение\\
$$x^{2} $$\\
Продифференцируем выражение\\
$$x$$\\
После дифференцирования получаем\\
$$1$$\\
Упростим выражение\\
$$1$$\\
Очевидно, данное выражение не нуждается в упрощении.\\
\newlineПосле дифференцирования получаем\\
$$2 \cdot x^{2 - 1}  \cdot 1$$\\
Упростим выражение\\
$$2 \cdot x^{2 - 1}  \cdot 1$$\\
Получим\\
$$2 \cdot x$$\\
Таким образом, производная функции
$$x^{2} $$равна
$$2 \cdot x$$\begin{figure}[h]
\centering
\includegraphics[width=0.8\linewidth]{images/plotimg137.png}
\caption{График производной}
\end{figure}
\newline
\newpage
\begin{center}
{\large \bf Построение касательной.}
\end{center}
Зная производную функции, можем построить касательную в точке $$x_0 = 1.000000$$
Уравнение касательной в точке $x_0$:
$$2 \cdot  ( x - 1 )  + 1$$\begin{figure}[h]
\centering
\includegraphics[width=0.8\linewidth]{images/plotimg138.png}
\caption{График касательной}
\end{figure}
\newline
\end{document}