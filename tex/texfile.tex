\documentclass[12pt,a4paper,fleqn]{article}
\usepackage[utf8]{inputenc}
\usepackage[russian]{babel}
\usepackage{graphicx}
\begin{document}
\begin{center}
{\Large \bf Методы дифференцирования функций в математическом анализе.}
\end{center}
Рассмотрим функцию
$$2^{x} $$Упростим выражение\\
$$2^{x} $$\\
Очевидно, данное выражение не нуждается в упрощении.\\
\newline\begin{figure}[h]
\centering
\includegraphics[width=0.8\linewidth]{images/plotimg190.png}
\caption{График функции}
\end{figure}
\newline
\newpage
\begin{center}
{\large \bf Нахождение производной.}
\end{center}
Продифференцируем выражение\\
$$2^{x} $$\\
Продифференцируем выражение\\
$$x$$\\
После дифференцирования получаем\\
$$1$$\\
Упростим выражение\\
$$1$$\\
Очевидно, данное выражение не нуждается в упрощении.\\
\newlineПосле дифференцирования получаем\\
$$ \ln 2 \cdot 2^{x}  \cdot 1$$\\
Упростим выражение\\
$$ \ln 2 \cdot 2^{x}  \cdot 1$$\\
Получим\\
$$0.693147 \cdot 2^{x} $$\\
Таким образом, производная функции
$$2^{x} $$равна
$$0.693147 \cdot 2^{x} $$\begin{figure}[h]
\centering
\includegraphics[width=0.8\linewidth]{images/plotimg191.png}
\caption{График производной}
\end{figure}
\newline
\newpage
\begin{center}
{\large \bf Построение касательной.}
\end{center}
Зная производную функции, можем построить касательную в точке$$x_0 = 1 $$
Значение функции в точке $x_0$ равно $2 $;\\
Значение производной в точке $x_0$ равно $1.386294$;\\
Уравнение касательной в точке $x_0$:
$$1.386294 \cdot  ( x - 1 )  + 2$$\begin{figure}[h]
\centering
\includegraphics[width=0.8\linewidth]{images/plotimg192.png}
\caption{График касательной}
\end{figure}
\newline
\newpage
\begin{center}
{\large \bf Разложение функции в ряд Маклорена.}
\end{center}
Значение функции при $x = 0$ равно $1$.\\
1-я производная функции равна$$0.693147 \cdot 2^{x} $$Значение 1-й производной при $x = 0$ равно $0.693147$.\\
Найдём  2-ю производную функции.
Продифференцируем выражение\\
$$0.693147 \cdot 2^{x} $$\\
Продифференцируем выражение\\
$$2^{x} $$\\
Продифференцируем выражение\\
$$x$$\\
После дифференцирования получаем\\
$$1$$\\
Упростим выражение\\
$$1$$\\
Очевидно, данное выражение не нуждается в упрощении.\\
\newlineПосле дифференцирования получаем\\
$$ \ln 2 \cdot 2^{x}  \cdot 1$$\\
Упростим выражение\\
$$ \ln 2 \cdot 2^{x}  \cdot 1$$\\
Получим\\
$$0.693147 \cdot 2^{x} $$\\
После дифференцирования получаем\\
$$0.693147 \cdot 0.693147 \cdot 2^{x} $$\\
Упростим выражение\\
$$0.693147 \cdot 0.693147 \cdot 2^{x} $$\\
Очевидно, данное выражение не нуждается в упрощении.\\
\newline2-я производная функции равна$$0.693147 \cdot 0.693147 \cdot 2^{x} $$Значение 2-й производной при $x = 0$ равно $0.480453$.\\
Найдём  3-ю производную функции.
Продифференцируем выражение\\
$$0.693147 \cdot 0.693147 \cdot 2^{x} $$\\
Продифференцируем выражение\\
$$0.693147 \cdot 2^{x} $$\\
Продифференцируем выражение\\
$$2^{x} $$\\
Продифференцируем выражение\\
$$x$$\\
После дифференцирования получаем\\
$$1$$\\
Упростим выражение\\
$$1$$\\
Очевидно, данное выражение не нуждается в упрощении.\\
\newlineПосле дифференцирования получаем\\
$$ \ln 2 \cdot 2^{x}  \cdot 1$$\\
Упростим выражение\\
$$ \ln 2 \cdot 2^{x}  \cdot 1$$\\
Получим\\
$$0.693147 \cdot 2^{x} $$\\
После дифференцирования получаем\\
$$0.693147 \cdot 0.693147 \cdot 2^{x} $$\\
Упростим выражение\\
$$0.693147 \cdot 0.693147 \cdot 2^{x} $$\\
Очевидно, данное выражение не нуждается в упрощении.\\
\newlineПосле дифференцирования получаем\\
$$0.693147 \cdot 0.693147 \cdot 0.693147 \cdot 2^{x} $$\\
Упростим выражение\\
$$0.693147 \cdot 0.693147 \cdot 0.693147 \cdot 2^{x} $$\\
Очевидно, данное выражение не нуждается в упрощении.\\
\newline3-я производная функции равна$$0.693147 \cdot 0.693147 \cdot 0.693147 \cdot 2^{x} $$Значение 3-й производной при $x = 0$ равно $0.333025$.\\
Разложение функции в ряд Маклорена до $x^3$:$$1 + 0.693147 \cdot x +  \frac{0.480453}{2}  \cdot x^{2}  +  \frac{0.333025}{6}  \cdot x^{3} $$Упростим выражение\\
$$1 + 0.693147 \cdot x +  \frac{0.480453}{2}  \cdot x^{2}  +  \frac{0.333025}{6}  \cdot x^{3} $$\\
Получим\\
$$1 + 0.693147 \cdot x + 0.240227 \cdot x^{2}  + 0.055504 \cdot x^{3} $$\\
\end{document}