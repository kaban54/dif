\documentclass[12pt,a4paper,fleqn]{article}
\usepackage[utf8]{inputenc}
\usepackage[russian]{babel}
\usepackage{graphicx}
\begin{document}
\begin{center}
{\Large \bf Методы дифференцирования функций в математическом анализе.}
\end{center}
Рассмотрим функцию
$$x^{2} $$Упростим выражение\\
$$x^{2} $$\\
Очевидно, данное выражение не нуждается в упрощении.\\
\newline\begin{figure}[h]
\centering
\includegraphics[width=0.8\linewidth]{images/plotimg223.png}
\caption{График функции}
\end{figure}
\newline
\newpage
\begin{center}
{\large \bf Нахождение производной.}
\end{center}
Продифференцируем выражение\\
$$x^{2} $$\\
Продифференцируем выражение\\
$$x$$\\
После дифференцирования получаем\\
$$1$$\\
Упростим выражение\\
$$1$$\\
Очевидно, данное выражение не нуждается в упрощении.\\
\newlineПосле дифференцирования получаем\\
$$2 \cdot x^{2 - 1}  \cdot 1$$\\
Упростим выражение\\
$$2 \cdot x^{2 - 1}  \cdot 1$$\\
Получим\\
$$2 \cdot x$$\\
Таким образом, производная функции
$$x^{2} $$равна
$$2 \cdot x$$\begin{figure}[h]
\centering
\includegraphics[width=0.8\linewidth]{images/plotimg224.png}
\caption{График производной}
\end{figure}
\newline
\newpage
\begin{center}
{\large \bf Построение касательной.}
\end{center}
Зная производную функции, можем построить касательную в точке$$x_0 = 1 $$
Значение функции в точке $x_0$ равно $1 $;\\
Значение производной в точке $x_0$ равно $2 $;\\
Уравнение касательной в точке $x_0$:
$$2 \cdot  ( x - 1 )  + 1$$\begin{figure}[h]
\centering
\includegraphics[width=0.8\linewidth]{images/plotimg225.png}
\caption{График касательной}
\end{figure}
\newline
\newpage
\begin{center}
{\large \bf Разложение функции в ряд Тейлора в точке $x_0 = $$1$.}
\end{center}
Значение функции при $x = x_0$ равно $1$.\\
1-я производная функции равна$$2 \cdot x$$Значение 1-й производной при $x = x_0$ равно $2$.\\
Найдём  2-ю производную функции.
Продифференцируем выражение\\
$$2 \cdot x$$\\
Продифференцируем выражение\\
$$x$$\\
После дифференцирования получаем\\
$$1$$\\
Упростим выражение\\
$$1$$\\
Очевидно, данное выражение не нуждается в упрощении.\\
\newlineПосле дифференцирования получаем\\
$$2 \cdot 1$$\\
Упростим выражение\\
$$2 \cdot 1$$\\
Получим\\
$$2$$\\
2-я производная функции равна$$2$$Значение 2-й производной при $x = x_0$ равно $2$.\\
Найдём  3-ю производную функции.
Продифференцируем выражение\\
$$2$$\\
После дифференцирования получаем\\
$$0$$\\
Упростим выражение\\
$$0$$\\
Очевидно, данное выражение не нуждается в упрощении.\\
\newline3-я производная функции равна$$0$$Значение 3-й производной при $x = x_0$ равно $0$.\\
Разложение функции в ряд Тейлора в точке $x_0$ до $x^3$:$$1 + 2 \cdot  ( x - 1 )  +  \frac{2}{2}  \cdot  ( x - 1 ) ^{2}  +  \frac{0}{6}  \cdot  ( x - 1 ) ^{3} $$Упростим выражение\\
$$1 + 2 \cdot  ( x - 1 )  +  \frac{2}{2}  \cdot  ( x - 1 ) ^{2}  +  \frac{0}{6}  \cdot  ( x - 1 ) ^{3} $$\\
Получим\\
$$1 + 2 \cdot  ( x - 1 )  +  ( x - 1 ) ^{2} $$\\
\end{document}