\documentclass[12pt,a4paper,fleqn]{article}
\usepackage[utf8]{inputenc}
\usepackage[russian]{babel}
\usepackage{graphicx}
\begin{document}
\begin{center}
{\Large \bf Методы дифференцирования функций в математическом анализе.}
\end{center}
Рассмотрим функцию
$$ (  \sin x ) ^{2}  + 3 \cdot  \cos  ( 5 \cdot x ) $$Упростим выражение\\
$$ (  \sin x ) ^{2}  + 3 \cdot  \cos  ( 5 \cdot x ) $$\\
Очевидно, данное выражение не нуждается в упрощении.\\
\newline\begin{figure}[h]
\centering
\includegraphics[width=0.8\linewidth]{images/plotimg214.png}
\caption{График функции}
\end{figure}
\newline
\newpage
\begin{center}
{\large \bf Нахождение производной.}
\end{center}
Продифференцируем выражение\\
$$ (  \sin x ) ^{2}  + 3 \cdot  \cos  ( 5 \cdot x ) $$\\
Продифференцируем выражение\\
$$3 \cdot  \cos  ( 5 \cdot x ) $$\\
Продифференцируем выражение\\
$$ \cos  ( 5 \cdot x ) $$\\
Продифференцируем выражение\\
$$5 \cdot x$$\\
Продифференцируем выражение\\
$$x$$\\
После дифференцирования получаем\\
$$1$$\\
Упростим выражение\\
$$1$$\\
Очевидно, данное выражение не нуждается в упрощении.\\
\newlineПосле дифференцирования получаем\\
$$5 \cdot 1$$\\
Упростим выражение\\
$$5 \cdot 1$$\\
Получим\\
$$5$$\\
После дифференцирования получаем\\
$$-1 \cdot  \sin  ( 5 \cdot x )  \cdot 5$$\\
Упростим выражение\\
$$-1 \cdot  \sin  ( 5 \cdot x )  \cdot 5$$\\
Очевидно, данное выражение не нуждается в упрощении.\\
\newlineПосле дифференцирования получаем\\
$$3 \cdot -1 \cdot  \sin  ( 5 \cdot x )  \cdot 5$$\\
Упростим выражение\\
$$3 \cdot -1 \cdot  \sin  ( 5 \cdot x )  \cdot 5$$\\
Очевидно, данное выражение не нуждается в упрощении.\\
\newlineПродифференцируем выражение\\
$$ (  \sin x ) ^{2} $$\\
Продифференцируем выражение\\
$$ \sin x$$\\
Продифференцируем выражение\\
$$x$$\\
После дифференцирования получаем\\
$$1$$\\
Упростим выражение\\
$$1$$\\
Очевидно, данное выражение не нуждается в упрощении.\\
\newlineПосле дифференцирования получаем\\
$$ \cos x \cdot 1$$\\
Упростим выражение\\
$$ \cos x \cdot 1$$\\
Получим\\
$$ \cos x$$\\
После дифференцирования получаем\\
$$2 \cdot  (  \sin x ) ^{2 - 1}  \cdot  \cos x$$\\
Упростим выражение\\
$$2 \cdot  (  \sin x ) ^{2 - 1}  \cdot  \cos x$$\\
Получим\\
$$2 \cdot  \sin x \cdot  \cos x$$\\
После дифференцирования получаем\\
$$2 \cdot  \sin x \cdot  \cos x + 3 \cdot -1 \cdot  \sin  ( 5 \cdot x )  \cdot 5$$\\
Упростим выражение\\
$$2 \cdot  \sin x \cdot  \cos x + 3 \cdot -1 \cdot  \sin  ( 5 \cdot x )  \cdot 5$$\\
Очевидно, данное выражение не нуждается в упрощении.\\
\newlineТаким образом, производная функции
$$ (  \sin x ) ^{2}  + 3 \cdot  \cos  ( 5 \cdot x ) $$равна
$$2 \cdot  \sin x \cdot  \cos x + 3 \cdot -1 \cdot  \sin  ( 5 \cdot x )  \cdot 5$$\begin{figure}[h]
\centering
\includegraphics[width=0.8\linewidth]{images/plotimg215.png}
\caption{График производной}
\end{figure}
\newline
\newpage
\begin{center}
{\large \bf Построение касательной.}
\end{center}
Зная производную функции, можем построить касательную в точке$$x_0 = 1 $$
Значение функции в точке $x_0$ равно $1.559060$;\\
Значение производной в точке $x_0$ равно $15.293162$;\\
Уравнение касательной в точке $x_0$:
$$15.293162 \cdot  ( x - 1 )  + 1.559060$$\begin{figure}[h]
\centering
\includegraphics[width=0.8\linewidth]{images/plotimg216.png}
\caption{График касательной}
\end{figure}
\newline
\newpage
\begin{center}
{\large \bf Разложение функции в ряд Маклорена.}
\end{center}
Значение функции при $x = 0$ равно $3$.\\
1-я производная функции равна$$2 \cdot  \sin x \cdot  \cos x + 3 \cdot -1 \cdot  \sin  ( 5 \cdot x )  \cdot 5$$Значение 1-й производной при $x = 0$ равно $0$.\\
Найдём  2-ю производную функции.
Продифференцируем выражение\\
$$2 \cdot  \sin x \cdot  \cos x + 3 \cdot -1 \cdot  \sin  ( 5 \cdot x )  \cdot 5$$\\
Продифференцируем выражение\\
$$3 \cdot -1 \cdot  \sin  ( 5 \cdot x )  \cdot 5$$\\
Продифференцируем выражение\\
$$-1 \cdot  \sin  ( 5 \cdot x )  \cdot 5$$\\
Продифференцируем выражение\\
$$ \sin  ( 5 \cdot x )  \cdot 5$$\\
Продифференцируем выражение\\
$$ \sin  ( 5 \cdot x ) $$\\
Продифференцируем выражение\\
$$5 \cdot x$$\\
Продифференцируем выражение\\
$$x$$\\
После дифференцирования получаем\\
$$1$$\\
Упростим выражение\\
$$1$$\\
Очевидно, данное выражение не нуждается в упрощении.\\
\newlineПосле дифференцирования получаем\\
$$5 \cdot 1$$\\
Упростим выражение\\
$$5 \cdot 1$$\\
Получим\\
$$5$$\\
После дифференцирования получаем\\
$$ \cos  ( 5 \cdot x )  \cdot 5$$\\
Упростим выражение\\
$$ \cos  ( 5 \cdot x )  \cdot 5$$\\
Очевидно, данное выражение не нуждается в упрощении.\\
\newlineПосле дифференцирования получаем\\
$$5 \cdot  \cos  ( 5 \cdot x )  \cdot 5$$\\
Упростим выражение\\
$$5 \cdot  \cos  ( 5 \cdot x )  \cdot 5$$\\
Очевидно, данное выражение не нуждается в упрощении.\\
\newlineПосле дифференцирования получаем\\
$$-1 \cdot 5 \cdot  \cos  ( 5 \cdot x )  \cdot 5$$\\
Упростим выражение\\
$$-1 \cdot 5 \cdot  \cos  ( 5 \cdot x )  \cdot 5$$\\
Очевидно, данное выражение не нуждается в упрощении.\\
\newlineПосле дифференцирования получаем\\
$$3 \cdot -1 \cdot 5 \cdot  \cos  ( 5 \cdot x )  \cdot 5$$\\
Упростим выражение\\
$$3 \cdot -1 \cdot 5 \cdot  \cos  ( 5 \cdot x )  \cdot 5$$\\
Очевидно, данное выражение не нуждается в упрощении.\\
\newlineПродифференцируем выражение\\
$$2 \cdot  \sin x \cdot  \cos x$$\\
Продифференцируем выражение\\
$$ \sin x \cdot  \cos x$$\\
Продифференцируем выражение\\
$$ \cos x$$\\
Продифференцируем выражение\\
$$x$$\\
После дифференцирования получаем\\
$$1$$\\
Упростим выражение\\
$$1$$\\
Очевидно, данное выражение не нуждается в упрощении.\\
\newlineПосле дифференцирования получаем\\
$$-1 \cdot  \sin x \cdot 1$$\\
Упростим выражение\\
$$-1 \cdot  \sin x \cdot 1$$\\
Получим\\
$$-1 \cdot  \sin x$$\\
Продифференцируем выражение\\
$$ \sin x$$\\
Продифференцируем выражение\\
$$x$$\\
После дифференцирования получаем\\
$$1$$\\
Упростим выражение\\
$$1$$\\
Очевидно, данное выражение не нуждается в упрощении.\\
\newlineПосле дифференцирования получаем\\
$$ \cos x \cdot 1$$\\
Упростим выражение\\
$$ \cos x \cdot 1$$\\
Получим\\
$$ \cos x$$\\
После дифференцирования получаем\\
$$ \cos x \cdot  \cos x +  \sin x \cdot -1 \cdot  \sin x$$\\
Упростим выражение\\
$$ \cos x \cdot  \cos x +  \sin x \cdot -1 \cdot  \sin x$$\\
Очевидно, данное выражение не нуждается в упрощении.\\
\newlineПосле дифференцирования получаем\\
$$2 \cdot  (  \cos x \cdot  \cos x +  \sin x \cdot -1 \cdot  \sin x ) $$\\
Упростим выражение\\
$$2 \cdot  (  \cos x \cdot  \cos x +  \sin x \cdot -1 \cdot  \sin x ) $$\\
Очевидно, данное выражение не нуждается в упрощении.\\
\newlineПосле дифференцирования получаем\\
$$2 \cdot  (  \cos x \cdot  \cos x +  \sin x \cdot -1 \cdot  \sin x )  + 3 \cdot -1 \cdot 5 \cdot  \cos  ( 5 \cdot x )  \cdot 5$$\\
Упростим выражение\\
$$2 \cdot  (  \cos x \cdot  \cos x +  \sin x \cdot -1 \cdot  \sin x )  + 3 \cdot -1 \cdot 5 \cdot  \cos  ( 5 \cdot x )  \cdot 5$$\\
Очевидно, данное выражение не нуждается в упрощении.\\
\newline2-я производная функции равна$$2 \cdot  (  \cos x \cdot  \cos x +  \sin x \cdot -1 \cdot  \sin x )  + 3 \cdot -1 \cdot 5 \cdot  \cos  ( 5 \cdot x )  \cdot 5$$Значение 2-й производной при $x = 0$ равно $-73$.\\
Найдём  3-ю производную функции.
Продифференцируем выражение\\
$$2 \cdot  (  \cos x \cdot  \cos x +  \sin x \cdot -1 \cdot  \sin x )  + 3 \cdot -1 \cdot 5 \cdot  \cos  ( 5 \cdot x )  \cdot 5$$\\
Продифференцируем выражение\\
$$3 \cdot -1 \cdot 5 \cdot  \cos  ( 5 \cdot x )  \cdot 5$$\\
Продифференцируем выражение\\
$$-1 \cdot 5 \cdot  \cos  ( 5 \cdot x )  \cdot 5$$\\
Продифференцируем выражение\\
$$5 \cdot  \cos  ( 5 \cdot x )  \cdot 5$$\\
Продифференцируем выражение\\
$$ \cos  ( 5 \cdot x )  \cdot 5$$\\
Продифференцируем выражение\\
$$ \cos  ( 5 \cdot x ) $$\\
Продифференцируем выражение\\
$$5 \cdot x$$\\
Продифференцируем выражение\\
$$x$$\\
После дифференцирования получаем\\
$$1$$\\
Упростим выражение\\
$$1$$\\
Очевидно, данное выражение не нуждается в упрощении.\\
\newlineПосле дифференцирования получаем\\
$$5 \cdot 1$$\\
Упростим выражение\\
$$5 \cdot 1$$\\
Получим\\
$$5$$\\
После дифференцирования получаем\\
$$-1 \cdot  \sin  ( 5 \cdot x )  \cdot 5$$\\
Упростим выражение\\
$$-1 \cdot  \sin  ( 5 \cdot x )  \cdot 5$$\\
Очевидно, данное выражение не нуждается в упрощении.\\
\newlineПосле дифференцирования получаем\\
$$5 \cdot -1 \cdot  \sin  ( 5 \cdot x )  \cdot 5$$\\
Упростим выражение\\
$$5 \cdot -1 \cdot  \sin  ( 5 \cdot x )  \cdot 5$$\\
Очевидно, данное выражение не нуждается в упрощении.\\
\newlineПосле дифференцирования получаем\\
$$5 \cdot 5 \cdot -1 \cdot  \sin  ( 5 \cdot x )  \cdot 5$$\\
Упростим выражение\\
$$5 \cdot 5 \cdot -1 \cdot  \sin  ( 5 \cdot x )  \cdot 5$$\\
Очевидно, данное выражение не нуждается в упрощении.\\
\newlineПосле дифференцирования получаем\\
$$-1 \cdot 5 \cdot 5 \cdot -1 \cdot  \sin  ( 5 \cdot x )  \cdot 5$$\\
Упростим выражение\\
$$-1 \cdot 5 \cdot 5 \cdot -1 \cdot  \sin  ( 5 \cdot x )  \cdot 5$$\\
Очевидно, данное выражение не нуждается в упрощении.\\
\newlineПосле дифференцирования получаем\\
$$3 \cdot -1 \cdot 5 \cdot 5 \cdot -1 \cdot  \sin  ( 5 \cdot x )  \cdot 5$$\\
Упростим выражение\\
$$3 \cdot -1 \cdot 5 \cdot 5 \cdot -1 \cdot  \sin  ( 5 \cdot x )  \cdot 5$$\\
Очевидно, данное выражение не нуждается в упрощении.\\
\newlineПродифференцируем выражение\\
$$2 \cdot  (  \cos x \cdot  \cos x +  \sin x \cdot -1 \cdot  \sin x ) $$\\
Продифференцируем выражение\\
$$ \cos x \cdot  \cos x +  \sin x \cdot -1 \cdot  \sin x$$\\
Продифференцируем выражение\\
$$ \sin x \cdot -1 \cdot  \sin x$$\\
Продифференцируем выражение\\
$$-1 \cdot  \sin x$$\\
Продифференцируем выражение\\
$$ \sin x$$\\
Продифференцируем выражение\\
$$x$$\\
После дифференцирования получаем\\
$$1$$\\
Упростим выражение\\
$$1$$\\
Очевидно, данное выражение не нуждается в упрощении.\\
\newlineПосле дифференцирования получаем\\
$$ \cos x \cdot 1$$\\
Упростим выражение\\
$$ \cos x \cdot 1$$\\
Получим\\
$$ \cos x$$\\
После дифференцирования получаем\\
$$-1 \cdot  \cos x$$\\
Упростим выражение\\
$$-1 \cdot  \cos x$$\\
Очевидно, данное выражение не нуждается в упрощении.\\
\newlineПродифференцируем выражение\\
$$ \sin x$$\\
Продифференцируем выражение\\
$$x$$\\
После дифференцирования получаем\\
$$1$$\\
Упростим выражение\\
$$1$$\\
Очевидно, данное выражение не нуждается в упрощении.\\
\newlineПосле дифференцирования получаем\\
$$ \cos x \cdot 1$$\\
Упростим выражение\\
$$ \cos x \cdot 1$$\\
Получим\\
$$ \cos x$$\\
После дифференцирования получаем\\
$$ \cos x \cdot -1 \cdot  \sin x +  \sin x \cdot -1 \cdot  \cos x$$\\
Упростим выражение\\
$$ \cos x \cdot -1 \cdot  \sin x +  \sin x \cdot -1 \cdot  \cos x$$\\
Очевидно, данное выражение не нуждается в упрощении.\\
\newlineПродифференцируем выражение\\
$$ \cos x \cdot  \cos x$$\\
Продифференцируем выражение\\
$$ \cos x$$\\
Продифференцируем выражение\\
$$x$$\\
После дифференцирования получаем\\
$$1$$\\
Упростим выражение\\
$$1$$\\
Очевидно, данное выражение не нуждается в упрощении.\\
\newlineПосле дифференцирования получаем\\
$$-1 \cdot  \sin x \cdot 1$$\\
Упростим выражение\\
$$-1 \cdot  \sin x \cdot 1$$\\
Получим\\
$$-1 \cdot  \sin x$$\\
Продифференцируем выражение\\
$$ \cos x$$\\
Продифференцируем выражение\\
$$x$$\\
После дифференцирования получаем\\
$$1$$\\
Упростим выражение\\
$$1$$\\
Очевидно, данное выражение не нуждается в упрощении.\\
\newlineПосле дифференцирования получаем\\
$$-1 \cdot  \sin x \cdot 1$$\\
Упростим выражение\\
$$-1 \cdot  \sin x \cdot 1$$\\
Получим\\
$$-1 \cdot  \sin x$$\\
После дифференцирования получаем\\
$$-1 \cdot  \sin x \cdot  \cos x +  \cos x \cdot -1 \cdot  \sin x$$\\
Упростим выражение\\
$$-1 \cdot  \sin x \cdot  \cos x +  \cos x \cdot -1 \cdot  \sin x$$\\
Очевидно, данное выражение не нуждается в упрощении.\\
\newlineПосле дифференцирования получаем\\
$$-1 \cdot  \sin x \cdot  \cos x +  \cos x \cdot -1 \cdot  \sin x +  \cos x \cdot -1 \cdot  \sin x +  \sin x \cdot -1 \cdot  \cos x$$\\
Упростим выражение\\
$$-1 \cdot  \sin x \cdot  \cos x +  \cos x \cdot -1 \cdot  \sin x +  \cos x \cdot -1 \cdot  \sin x +  \sin x \cdot -1 \cdot  \cos x$$\\
Очевидно, данное выражение не нуждается в упрощении.\\
\newlineПосле дифференцирования получаем\\
$$2 \cdot  ( -1 \cdot  \sin x \cdot  \cos x +  \cos x \cdot -1 \cdot  \sin x +  \cos x \cdot -1 \cdot  \sin x +  \sin x \cdot -1 \cdot  \cos x ) $$\\
Упростим выражение\\
$$2 \cdot  ( -1 \cdot  \sin x \cdot  \cos x +  \cos x \cdot -1 \cdot  \sin x +  \cos x \cdot -1 \cdot  \sin x +  \sin x \cdot -1 \cdot  \cos x ) $$\\
Очевидно, данное выражение не нуждается в упрощении.\\
\newlineПосле дифференцирования получаем\\
$$2 \cdot  ( -1 \cdot  \sin x \cdot  \cos x +  \cos x \cdot -1 \cdot  \sin x +  \cos x \cdot -1 \cdot  \sin x +  \sin x \cdot -1 \cdot  \cos x )  + 3 \cdot -1 \cdot 5 \cdot 5 \cdot -1 \cdot  \sin  ( 5 \cdot x )  \cdot 5$$\\
Упростим выражение\\
$$2 \cdot  ( -1 \cdot  \sin x \cdot  \cos x +  \cos x \cdot -1 \cdot  \sin x +  \cos x \cdot -1 \cdot  \sin x +  \sin x \cdot -1 \cdot  \cos x )  + 3 \cdot -1 \cdot 5 \cdot 5 \cdot -1 \cdot  \sin  ( 5 \cdot x )  \cdot 5$$\\
Очевидно, данное выражение не нуждается в упрощении.\\
\newline3-я производная функции равна$$2 \cdot  ( -1 \cdot  \sin x \cdot  \cos x +  \cos x \cdot -1 \cdot  \sin x +  \cos x \cdot -1 \cdot  \sin x +  \sin x \cdot -1 \cdot  \cos x )  + 3 \cdot -1 \cdot 5 \cdot 5 \cdot -1 \cdot  \sin  ( 5 \cdot x )  \cdot 5$$Значение 3-й производной при $x = 0$ равно $0$.\\
Разложение функции в ряд Маклорена до $x^3$:$$3 + 0 \cdot x +  \frac{-73}{2}  \cdot x^{2}  +  \frac{0}{6}  \cdot x^{3} $$Упростим выражение\\
$$3 + 0 \cdot x +  \frac{-73}{2}  \cdot x^{2}  +  \frac{0}{6}  \cdot x^{3} $$\\
Получим\\
$$3 + -36.500000 \cdot x^{2} $$\\
\end{document}